\documentclass[11pt]{article}
\usepackage[a4paper, left=1in, right=1in, top=1in, bottom=1in]{geometry}
\newcommand*{\authorfont}{\fontfamily{phv}\selectfont}
\usepackage[english, greek, main=english]{babel}
\usepackage[]{mathptmx}
\usepackage[LGR,T1]{fontenc}
\usepackage[utf8]{inputenc}
\usepackage[page,toc,titletoc,title]{appendix}  % appendix, table of contents
\usepackage{enumitem}  % package for adjusting itemize sep
\usepackage{amssymb}  % extra math fonts, like \mathbb, \mathcal
\usepackage{amsmath}
\usepackage{amsthm}
\theoremstyle{definition}
\newtheorem{definition}{Definition}
\newtheorem{example}[definition]{Example}  % example numbering depends on definition
\usepackage{abstract}
\renewcommand{\abstractname}{}    % clear the title
\renewcommand{\absnamepos}{empty} % originally center
\usepackage{tikz}
\usepackage{pgfplots}
\usepackage{xcolor}
\pgfplotsset{compat=1.16}
\usepackage{tcolorbox}
\usepackage{listings}
\usepackage{subcaption}
\usepackage{booktabs}
\usepackage{varwidth}
\usepackage[export]{adjustbox}
\usetikzlibrary{arrows}

\renewenvironment{abstract}
 {{%
    \setlength{\leftmargin}{0mm}
    \setlength{\rightmargin}{\leftmargin}%
  }%
  \relax}
 {\endlist}

\makeatletter
\def\@maketitle{%
  \newpage
%  \null
%  \vskip 2em%
%  \begin{center}%
  \let \footnote \thanks
    {\fontsize{18}{20}\selectfont\raggedright  \setlength{\parindent}{0pt} \@title \par}%
}
%\fi
\makeatother




\setcounter{secnumdepth}{0}

\usepackage{color}
\usepackage{fancyvrb}
\newcommand{\VerbBar}{|}
\newcommand{\VERB}{\Verb[commandchars=\\\{\}]}
\DefineVerbatimEnvironment{Highlighting}{Verbatim}{commandchars=\\\{\}}
% Add ',fontsize=\small' for more characters per line
\usepackage{framed}
\definecolor{shadecolor}{RGB}{248,248,248}
\newenvironment{Shaded}{\begin{snugshade}}{\end{snugshade}}
\newcommand{\AlertTok}[1]{\textcolor[rgb]{0.94,0.16,0.16}{#1}}
\newcommand{\AnnotationTok}[1]{\textcolor[rgb]{0.56,0.35,0.01}{\textbf{\textit{#1}}}}
\newcommand{\AttributeTok}[1]{\textcolor[rgb]{0.77,0.63,0.00}{#1}}
\newcommand{\BaseNTok}[1]{\textcolor[rgb]{0.00,0.00,0.81}{#1}}
\newcommand{\BuiltInTok}[1]{#1}
\newcommand{\CharTok}[1]{\textcolor[rgb]{0.31,0.60,0.02}{#1}}
\newcommand{\CommentTok}[1]{\textcolor[rgb]{0.56,0.35,0.01}{\textit{#1}}}
\newcommand{\CommentVarTok}[1]{\textcolor[rgb]{0.56,0.35,0.01}{\textbf{\textit{#1}}}}
\newcommand{\ConstantTok}[1]{\textcolor[rgb]{0.00,0.00,0.00}{#1}}
\newcommand{\ControlFlowTok}[1]{\textcolor[rgb]{0.13,0.29,0.53}{\textbf{#1}}}
\newcommand{\DataTypeTok}[1]{\textcolor[rgb]{0.13,0.29,0.53}{#1}}
\newcommand{\DecValTok}[1]{\textcolor[rgb]{0.00,0.00,0.81}{#1}}
\newcommand{\DocumentationTok}[1]{\textcolor[rgb]{0.56,0.35,0.01}{\textbf{\textit{#1}}}}
\newcommand{\ErrorTok}[1]{\textcolor[rgb]{0.64,0.00,0.00}{\textbf{#1}}}
\newcommand{\ExtensionTok}[1]{#1}
\newcommand{\FloatTok}[1]{\textcolor[rgb]{0.00,0.00,0.81}{#1}}
\newcommand{\FunctionTok}[1]{\textcolor[rgb]{0.00,0.00,0.00}{#1}}
\newcommand{\ImportTok}[1]{#1}
\newcommand{\InformationTok}[1]{\textcolor[rgb]{0.56,0.35,0.01}{\textbf{\textit{#1}}}}
\newcommand{\KeywordTok}[1]{\textcolor[rgb]{0.13,0.29,0.53}{\textbf{#1}}}
\newcommand{\NormalTok}[1]{#1}
\newcommand{\OperatorTok}[1]{\textcolor[rgb]{0.81,0.36,0.00}{\textbf{#1}}}
\newcommand{\OtherTok}[1]{\textcolor[rgb]{0.56,0.35,0.01}{#1}}
\newcommand{\PreprocessorTok}[1]{\textcolor[rgb]{0.56,0.35,0.01}{\textit{#1}}}
\newcommand{\RegionMarkerTok}[1]{#1}
\newcommand{\SpecialCharTok}[1]{\textcolor[rgb]{0.00,0.00,0.00}{#1}}
\newcommand{\SpecialStringTok}[1]{\textcolor[rgb]{0.31,0.60,0.02}{#1}}
\newcommand{\StringTok}[1]{\textcolor[rgb]{0.31,0.60,0.02}{#1}}
\newcommand{\VariableTok}[1]{\textcolor[rgb]{0.00,0.00,0.00}{#1}}
\newcommand{\VerbatimStringTok}[1]{\textcolor[rgb]{0.31,0.60,0.02}{#1}}
\newcommand{\WarningTok}[1]{\textcolor[rgb]{0.56,0.35,0.01}{\textbf{\textit{#1}}}}

\usepackage{graphicx,grffile}
\makeatletter
\def\maxwidth{\ifdim\Gin@nat@width>\linewidth\linewidth\else\Gin@nat@width\fi}
\def\maxheight{\ifdim\Gin@nat@height>\textheight\textheight\else\Gin@nat@height\fi}
\makeatother
% Scale images if necessary, so that they will not overflow the page
% margins by default, and it is still possible to overwrite the defaults
% using explicit options in \includegraphics[width, height, ...]{}
\setkeys{Gin}{width=\maxwidth,height=\maxheight,keepaspectratio}

\usepackage{titlesec}

\titleformat*{\section}{\normalsize\bfseries}
\titleformat*{\subsection}{\normalsize\itshape}
\titleformat*{\subsubsection}{\normalsize\itshape}
\titleformat*{\paragraph}{\normalsize\itshape}
\titleformat*{\subparagraph}{\normalsize\itshape}

\usepackage{natbib}
\usepackage[strings]{underscore} % protect underscores in most circumstances

\newtheorem{hypothesis}{Hypothesis}
\usepackage{setspace}

% set default figure placement to htbp
\makeatletter
\def\fps@figure{htbp}
\makeatother

\usepackage{hyperref}

% move the hyperref stuff down here, after header-includes, to allow for - \usepackage{hyperref}

\makeatletter
\@ifpackageloaded{hyperref}{}{%
\ifxetex
  \PassOptionsToPackage{hyphens}{url}\usepackage[setpagesize=false, % page size defined by xetex
              unicode=false, % unicode breaks when used with xetex
              xetex]{hyperref}
\else
  \PassOptionsToPackage{hyphens}{url}\usepackage[draft,unicode=true]{hyperref}
\fi
}

\@ifpackageloaded{color}{
    \PassOptionsToPackage{usenames,dvipsnames}{color}
}{%
    \usepackage[usenames,dvipsnames]{color}
}
\makeatother
\hypersetup{breaklinks=true,
            bookmarks=true,
            pdfauthor={Steven V. Miller (Clemson University)},
             pdfkeywords = {pandoc, r markdown, knitr},  
            pdftitle={A Pandoc Markdown Article Starter and Template},
            colorlinks=true,
            citecolor=blue,
            urlcolor=blue,
            linkcolor=magenta,
            pdfborder={0 0 0}}
\urlstyle{same}  % don't use monospace font for urls

% Add an option for endnotes. -----


% add tightlist ----------
\providecommand{\tightlist}{%
\setlength{\itemsep}{0pt}\setlength{\parskip}{0pt}}

% add some other packages ----------

% \usepackage{multicol}
% This should regulate where figures float
% See: https://tex.stackexchange.com/questions/2275/keeping-tables-figures-close-to-where-they-are-mentioned
\usepackage[section]{placeins}


% Title and author

\title{Hands-On Data Analysis for ININ Using R}

\author{\Large Prof. Dr. Cornelia Storz, M.Sc. Fei (Michael) Wang\vspace{0.05in} \newline\normalsize\emph{Management and Microeconomics, Goethe-Universität Frankfurt}  }

\date{}

% colback=white,
% colframe=white!50!black, 
% title = {Sample space: countable or uncountable?}, 
% fonttitle = \large

% set tcolorbox environment
\newtcolorbox{mybox}[1]
{
  colframe = white!50!black,
  colback  = white,
  fonttitle = \large, 
  title    = {#1}
}

\xdefinecolor{gray}{rgb}{0.4,0.4,0.4}
\xdefinecolor{blue}{RGB}{58,95,205}% R's royalblue3; #3A5FCD

\lstset{% setup listings
	language=R,% set programming language
	basicstyle=\ttfamily\small,% basic font style
	keywordstyle=\color{blue},% keyword style
        commentstyle=\color{gray},% comment style
	numbers=left,% display line numbers on the left side
	numberstyle=\scriptsize,% use small line numbers
	numbersep=10pt,% space between line numbers and code
	tabsize=3,% sizes of tabs
  commentstyle=\itshape\color{green!50!gray},
  frame=single,% single frame around code
	showstringspaces=false,% do not replace spaces in strings by a certain character
	captionpos=b,% positioning of the caption below
        breaklines=true,% automatic line breaking
        escapeinside={(*}{*)},% escaping to LaTeX
        fancyvrb=true,% verbatim code is typset by listings
        extendedchars=false,% prohibit extended chars (chars of codes 128--255)
        literate={"}{{\texttt{"}}}1{<-}{{$\bm\leftarrow$}}1{<<-}{{$\bm\twoheadleftarrow$}}1
        {~}{{$\bm\sim$}}1{<=}{{$\bm\le$}}1{>=}{{$\bm\ge$}}1{!=}{{$\bm\neq$}}1{^}{{$^{\bm\wedge}$}}1,% item to replace, text, length of chars
        alsoletter={.<-},% becomes a letter
        alsoother={$},% becomes other
        otherkeywords={!=, ~, $, \&, \%/\%, \%*\%, \%\%, <-, <<-, /},% other keywords
        deletekeywords={c}% remove keywords
}



%----------------------------begin document--------------------------------%

\begin{document}
	
% \pagenumbering{arabic}% resets `page` counter to 1 
%
% \maketitle

{% \usefont{T1}{pnc}{m}{n}
\setlength{\parindent}{0pt}
\thispagestyle{plain}
{\fontsize{18}{20}\selectfont\raggedright 
\maketitle  % title \par  

}

{
   \vskip 13.5pt\relax \normalsize\fontsize{11}{12} 
   \authorfont Prof. Dr. Cornelia Storz, M.Sc. Fei (Michael) Wang \\
   \\ 
    \emph{\small Management and Microeconomics, Goethe-Universität Frankfurt}   
}

}



\begin{abstract}
    \hbox{\vrule height .2pt width 39.14pc}
    \vskip 8.5pt % \small 

\noindent This document was prepared for students who are taking ININ course and
planning to take the exam. It is a collection of notes and codes for the
course. The notes are based on the tutorials we had in the course. I am trying
to make it concise and easy to understand. I hope it can help you to review
the course and prepare for the exam. \textit{We are living in a very noisy world,
therefore let's keep it simple and clear.} I setup a challenge for myself to
deliver a clear and concise review notes within 15 pages. This brings the trade-off,
which means some figures and tables are not included in the notes. Therefore,
you have to run the codes to see the results.  

\vskip 6pt 

\noindent I hope you enjoy reading it. I also hope you will have this notes with you
whenever you want to do some data analysis. If one day, you still refer to this
notes and find it still useful, I would be very happy to hear that.

 

\vskip 8.5pt \noindent \emph{Keywords}: econometrics, data analysis, regression models, 
empirical research, innovation, management  \par
\hbox{\vrule height .2pt width 39.14pc}

\end{abstract}

\vskip -8.5pt

 % removetitleabstract

\noindent  

% table of contents
\pagenumbering{roman}
\hypersetup{linkcolor=black}
\setcounter{tocdepth}{3}
\tableofcontents


\newpage
\pagenumbering{arabic}
\setcounter{page}{1}
\section{1 Introduction}

All statistical or econometric or machine learning models are based on 
the following assumptions:

\begin{itemize}
\tightlist
\item there are something we know - \textbf{data}
\item and something we don't know - \textbf{error} \(\epsilon\).
\end{itemize}

In summary, according to confucius, \textit{to know what we know and what we do not know}, 
that is called \textbf{wisdom}. Or like Plato said, \textit{I know that I know nothing}.
To help you to review the course, the notes will be organized as follows:

\begin{enumerate}
\tightlist 
  \item \textbf{Data}: using \texttt{data.table} to get familiar with the data
  \item \textbf{Simple linear regression}: how to estimate a simple linear regression model, how to interpret the results
  \item \textbf{Multiple linear regression}: how to estimate a multiple linear regression model, how to interpret the results, how to test the model
  \item \textbf{Introduction to logistic regression}: why do we need logistic regression
  \item \textbf{Data manipulation}: will not be tested in the exam, but it is very useful for your future work or research
\end{enumerate}


\section{2 Introduction to Data and \texttt{data.table}}



Broadly speaking, there are two kinds of data: \textbf{structured data} and 
\textbf{unstructured data}. 
Structured data is data that has a structure, such as a table, 
whereas unstructured data is data that does not have a structure, 
such as a text file. In this course, we focus on structured data. 
This means all the data we will use look like tables, such as the following one:


\begin{figure}
  \centering
  \includegraphics[width=0.9\textwidth]{../drawio/R-data-table-illustration.png}
\end{figure}













\end{document}